\documentclass{article}
\usepackage{geometry}
\usepackage{natbib}
\usepackage{amssymb}
\usepackage{amsmath}
\usepackage{graphicx}
\usepackage{hyperref}
\usepackage{textcomp}

\geometry{
	a4paper,
	total={170mm,257mm},
	left=30mm,
	top=30mm,
	bottom=20mm,
	right=20mm
}
\title{
	Projeto de Iniciação Científica - Relatório 1\\
	Visualização Unidimensional do Espaço Tridimensional \\
	\large Prólogo para Visualização Bidimensional do Espaço Quadridimensional
}
\date{2023-06-28}
\author{Paulo Roberto Rodrigues da Silva Filho\\ \small Felipe Acker (Orientador)}

\newcommand\R{\mathbb{R}}

\begin{document}
	\renewcommand{\figurename}{Figura}
	\graphicspath{ {./imagens/} }
	\maketitle
	\tableofcontents
	
	\section{Introdução}
	
	\paragraph{}
	Atualmente, os algoritmos de Ray-Tracing, visualização e radiosidade, para projeção de espaços 3D em 2D estão dominados e tecnologicamente avançados, já tendo mesmo implementações em Hadware, através de placas de vídeo 3D \textbf{[adicionar referências]}. Entretanto, tais técnicas implementam a representação do espaço tridimensional no espaço bidimensional, reduzindo apenas uma dimensão de representação, e apenas para o caso particular de $\R^3$. Entratanto, não há tecnologias consistentes que permitam a redução de $\R^3$ para $\R$, ou de $\R^4$ para $\R^2$ - objetivo final desse projeto.
	
	\paragraph{}
	Assim, foi necessário desenvolver a tecnologia e os algoritmos para essas representações da estaca zero. Esse relatório visa apresentar os cálculos necessários para prover tal renderização de objetos tridimensionais, em projeção unidimensional, utilizando o modelamento físico de olho, apresentado na proposta e na renderização de objetos quadridimensionais em projeção bidimensional.
	
	\paragraph{}
	O processo de irradiação luminosa e focalização é estendido do caso tridimensional para o caso quadridimensional, enquanto o processo de captura de imagem é reduzido do caso bidimensional para o caso unidimensional, nas projeções de $\R^3$ para $\R$, ou do caso tridimensional para o caso bidimensional, nas projeções de $\R^4$ para $\R^2$.
	
	\paragraph{}
	Em um primeiro momento, é entendido o processo de renderização de objetos bidimensionais em uma tela monodimensional, com a utilização de sombras, com o cálculo de radiosidade, mas sem a utilização completa de \textit{ray-tracing}, ou seja, não serão consideradas superfícies espelhadas. A partir daí, esse algoritmo é estendido para a renderização de objetos tridimensionais em tela monodimensional - que é o objetivo planejado para esse projeto - e, por último, esse algoritmo é estendido para a renderização de objetos quadridimensionais em telas bidimensionals, o que já seria o objetivo final do projeto. 
	
	\section{Modelo Mínimo - Ambiente Bidimensional em Tela Monodimensional}
	
	\paragraph{}
	O modelo mínimo é a renderização de objetos bidimensionais em uma tela unidimensional. Esse modelo de renderização é \textbf{inferior} ao que já é implementado atualmente em software e em hardware, pelas placas 3D, mas é a base do modelo de renderização quadridimensional - portanto, entendê-lo é fundamental para a implementação de modelos de renderização que fujam do padronizado, que é a Renderização de Ambientes Tridimensionais em Telas Bidimensionais.
	
	\paragraph{}
	Para haver renderização, é necessário haver \textbf{iluminação} e \textbf{captura}. A captura é feita pelo olho, é já é uma característica da renderização assumida por definição. A iluminação, apesar de obrigatória para a renderização, não é imediatamente considerada pelo senso comum. Entretanto, renderização sem um modelo de iluminação adequada apresentaria figuras achatadas e sem volume aparente, que permitiria discernir sobre as formas representadas na renderização.
	
	\paragraph{}
	Usamos dois modelos de iluminação: \textbf{(1)} Iluminação Paralela, adequada para ambientes externos e \textbf{(2)} Iluminação radial, adequada para ambientes internos. Em ambos os casos adicionamos um componente de iluminação difusa. Não é apresentado o algoritmo de \textit{tinting}, então todas as fontes luminosas são brancas e não é apresentado o algoritmo para a utilização de mais de uma fonte luminosa. Essas omissões são corrigidas posteriormente, em uma seção específica para elas.
	
	\subsection{Iluminação Paralela}
	
	\paragraph{}
	\textbf{Continua...}
	
	\subsection{Iluminação Radial}
	
	\paragraph{}
	\textbf{Continua...}
	
	\section{Modelo Ampliado (1) - Ambiente Tridimensional em Tela Monodimensional}
	
	\paragraph{}
	\textbf{Continua...}
	
	\section{Modelo Ampliado (2) - Ambiente Quadridimensional em Tela bidimensional}
	
	\paragraph{}
	\textbf{Continua...}
	
	\section{Modelo Ampliado (3) - Ambiente N-Dimensional em Tela M-dimensional}
	
	\paragraph{}
	\textbf{Continua...}
	
	\section{Múltiplas fontes de Luz e \textit{tinting}}
	
	\paragraph{}
	Essa seção apresenta a implementação da importante omissão do uso de múltiplas fontes de luz e do \textit{tinting}. O \textit{tinting} é a aplicação de fontes de luz coloridas, que alteram as cores dos objetos iluminados segundo regras específicas diferentes do \textit{alpha-blending} utilizado nas renderizações apresentadas acima. Já o uso de multiplas fontes de luz provoca um grande aumento de complexidade para a renderização de cada pixel da tela, ainda mais, considerando que cada fonte de luz pode ter uma cor diferente, alterando as regras de \textit{tinting}.
	
	\textbf{Continua...}
	
	\section{Conclusões e Próximos Passos}
	
	\paragraph{}
	\textbf{Continua...}

	
\end{document}