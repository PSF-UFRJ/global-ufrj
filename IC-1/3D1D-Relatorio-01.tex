\documentclass{article}
\usepackage{geometry}
\usepackage{natbib}
\usepackage{amssymb}
\usepackage{amsmath}
\usepackage{graphicx}
\usepackage{hyperref}
\usepackage{textcomp}

\geometry{
	a4paper,
	total={170mm,257mm},
	left=30mm,
	top=30mm,
	bottom=20mm,
	right=20mm
}
\title{
	Projeto de Iniciação Científica - Relatório 1\\
	Visualização Unidimensional do Espaço Tridimensional \\
	\large Prólogo para Visualização Bidimensional do Espaço Quadridimensional
}
\date{2023-06-28}
\author{Paulo Roberto Rodrigues da Silva Filho\\ \small Felipe Acker (Orientador)}

\newcommand\R{\mathbb{R}}

\begin{document}
	\renewcommand{\figurename}{Figura}
	\graphicspath{ {./imagens/} }
	\maketitle
	\tableofcontents
	
	\section{Introdução}
	
	\paragraph{}
	Atualmente, os algoritmos de Ray-Tracing, visualização e radiosidade, para projeção de espaços 3D em 2D estão dominados e tecnologicamente avançados, já tendo mesmo implementações em Hadware, através de placas de vídeo 3D \textbf{[adicionar referências]}. Entretanto, tais técnicas implementam a representação do espaço tridimensional no espaço bidimensional, reduzindo apenas uma dimensão de representação, e apenas para o caso particular de $\R^3$. Como não há tecnologias que permitam a redução de $\R^3$ para $\R$, ou de $\R^4$ para $\R^2$ - objetivo final desse projeto.
	
	\paragraph{}
	Assim, foi necessário desenvolver a tecnologia e os algoritmos para essas representações da estaca zero. Esse relatório visa apresentar os cálculos necessários para prover tal renderização de objetos tridimensionais, em projeção unidimensional, utilizando o modelamento físico de olho, apresentado na proposta e na renderização de objetos quadridimensionais em projeção bidimensional.
	
	\paragraph{}
	O processo de irradiação luminosa e focalização é estendido do caso tridimensional para o caso quadridimensional, enquanto o processo de captura de imagem é reduzido do caso bidimensional para o caso unidimensional, nas projeções de $\R^3$ para $\R$, ou do caso tridimensional para o caso bidimensional, nas projeções de $\R^4$ para $\R^2$.
	
	
\end{document}